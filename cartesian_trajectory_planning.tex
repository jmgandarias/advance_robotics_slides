%% This Beamer template is based on the one found here: https://github.com/sanhacheong/stanford-beamer-presentation, and edited to be used for Stanford ARM Lab

\documentclass[10pt, aspectratio=169]{beamer}
%\mode<presentation>{}

\usepackage{media9}
\usepackage{amssymb,amsmath,amsthm,enumerate}
\usepackage[utf8]{inputenc}
\usepackage{array}
\usepackage[parfill]{parskip}
\usepackage{graphicx}
\usepackage{caption}
\usepackage{subcaption}
\usepackage{bm}
\usepackage{amsfonts,amscd}
\usepackage[]{units}
\usepackage{listings}
\usepackage{multicol}
\usepackage{multirow}
\usepackage{tcolorbox}
\usepackage{physics}

% Enable colored hyperlinks
\hypersetup{
    colorlinks=true,
    citecolor=uma_pink,
    linkcolor=uma_blue_light,
    filecolor=uma_blue_water,      
    urlcolor=uma_blue_light,
    pdftitle={Overleaf Example},
    pdfpagemode=FullScreen,
}

% Select normal math font
\usefonttheme[onlymath]{serif}

\renewcommand{\thebibliography}{\textcolor{uma_blue_water}{\arabic{bibliography}}}
\renewcommand{\thefigure}{\textcolor{uma_blue_water}{\arabic{figure}}}
\renewcommand{\figurename}{\textcolor{uma_blue_water}{Fig.}}
\renewcommand{\thesubfigure}{\textcolor{uma_blue_water}{\alph{subfigure}}}
\renewcommand{\thetable}{\textcolor{uma_blue_water}{\arabic{table}}}
\renewcommand{\tablename}{\textcolor{uma_blue_water}{Table}}


% The following three lines are for crossmarks & checkmarks
\usepackage{pifont}% http://ctan.org/pkg/pifont
\newcommand{\cmark}{\ding{51}}%
\newcommand{\xmark}{\ding{55}}%

% Numbered captions of tables, pictures, etc.
\setbeamertemplate{caption}[numbered]

%\usepackage[superscript,biblabel]{cite}
\usepackage{algorithm2e}
\renewcommand{\thealgocf}{}

% Bibliography settings
\usepackage[style=ieee]{biblatex}
\setbeamertemplate{bibliography item}{\insertbiblabel}
\addbibresource{references.bib}

% Glossary entries
\usepackage[acronym]{glossaries}
\newacronym{ML}{ML}{machine learning}
\newacronym{HRI}{HRI}{human-robot interactions}
\newacronym{RNN}{RNN}{Recurrent Neural Network}
\newacronym{LSTM}{LSTM}{Long Short-Term Memory}


\theoremstyle{remark}
\newtheorem*{remark}{Remark}
\theoremstyle{definition}

\newcommand{\empy}[1]{{\color{uma_blue_dark}\emph{#1}}}
\newcommand{\empr}[1]{{\color{uma_blue_dark}\emph{#1}}}
\newcommand{\examplebox}[2]{
\begin{tcolorbox}[colframe=uma_blue_dark,colback=uma_gray_light,title=#1]
#2
\end{tcolorbox}}

\usetheme{Uma} 
\input{./style_files_uma/my_beamer_defs.sty}
\logo{\includegraphics[height=0.8cm]{./style_files_uma/logo_uma_negativo}\hspace{0.1cm}}

% commands to relax beamer and subfig conflicts
% see here: https://tex.stackexchange.com/questions/426088/texlive-pretest-2018-beamer-and-subfig-collide
\makeatletter
\let\@@magyar@captionfix\relax
\makeatother

\title[\href{https://jmgandarias.com}{\textcolor{white}{jmgandarias.com}}]{Trajectory Planning in the Operational Space}

%\subtitle{Subtitle Of Presentation}

%\beamertemplatenavigationsymbolsempty

\begin{document}

\author[Systems Engineering and Automation]{
	\large
	Juan M. Gandarias\\
    \footnotesize \href{mailto:jmgandarias@uma.es}{jmgandarias@uma.es}
}

\institute{
	\textcolor{uma_gray_dark}{
    Systems Engineering and Automation Department\\
	University of Malaga\\
    \href{https://www.uma.es/imech/}{IMECH.UMA}}
 	\vskip 5pt
    % \small{\date{\today}}
 %    \begin{figure}
	% 	\centering
	% 	\begin{subfigure}[t]{0.5\textwidth}
	% 		\centering
	% 		\includegraphics[height=1.5cm]{./style_files_uma/logo_uma}
	% 	\end{subfigure}%
	% 	~
	% 	\begin{subfigure}[t]{0.5\textwidth}
	% 		\centering
	% 		\includegraphics[height=0.33in]{./images/arm_lab_logo_with_title_small_adj_6.png}
	% 	\end{subfigure}
	% \end{figure}
}


\date{\today}

\begin{noheadline}
\begin{frame}
    \maketitle
    \vspace{-1cm}
    \begin{figure}
		\centering
		\includegraphics[height=1.5cm]{./style_files_uma/logo_uma}
        \hspace{10cm}
        \includegraphics[height=1.4cm]{./style_files_uma/logo_isa}
	\end{figure}
 %    \begin{figure}
	% 	\centering
	% 	\begin{subfigure}[t]{0.5\textwidth}
	% 		\centering
	% 		\includegraphics[height=1.5cm]{./style_files_uma/logo_uma}
	% 	\end{subfigure}%
	% 	~
	% 	\begin{subfigure}[t]{0.5\textwidth}
	% 		\centering
	% 		\includegraphics[height=0.33in]{./images/arm_lab_logo_with_title_small_adj_6.png}
	% 	\end{subfigure}
	% \end{figure}
 \end{frame}
\end{noheadline}



\setbeamertemplate{itemize items}[default]
\setbeamertemplate{itemize subitem}[circle]

\begin{frame}
	\frametitle{Overview} % Table of contents slide, comment this block out to remove it
	\tableofcontents % Throughout your presentation, if you choose to use \section{} and \subsection{} commands, these will automatically be printed on this slide as an overview of your presentation
\end{frame}



\section{Introduction to Trajectory Planning}

\begin{frame}[allowframebreaks]
\frametitle{Definitions}
	\begin{itemize}
    
	    \item \textbf{Trajectory Planning:} To generate the references to the motion control system of a robotic manipulator.
        \begin{itemize}
            \item References are computed from a polynomial that adjusts to the desired trajectory.
            \item During the whole motion from the initial to the last pose the actuators are not saturated (smooth trajectories).
        \end{itemize}
        
    \item \textbf{Path:} Sequence of points in the joint or operational (Cartesian) space. 
    
    \item \textbf{Trajectory:} Path with a temporal law in terms of velocity or acceleration (path + schedule) \cite{paths_and_trajectories}.
    
    \item \textbf{Joint space:} Set of all possible positions and orientations of a robot's joints (a.k.a. configuration space).
    
    $\mathbf{q} = [q_1, ..., q_n ]^T \in \mathbb{R}^n$, $n \equiv \textrm{number of joint DoFs}$ 

    \framebreak
    
    \item \textbf{Operational space:} Space in which the robot's end-effector operates. Defined by the EE pose (a.k.a. task space).

    \item \textbf{Cartesian space:} Operational space described by the Cartesian coordinates and orientation angles.
    
    $\mathbf{x} = [\underbrace{x, y, z}_{position}, \underbrace{\alpha_x, \alpha_y, \alpha_z}_{orientation}] \in \mathbb{R}^6$.

    \item \textbf{Kinematics:} Study of motion without forces

    \begin{itemize}
        \item Forward kinematics: $\mathbf{x} = f(\mathbf{q})$
        \item Inverse kinematics: $\mathbf{q} = g(\mathbf{x})$
        \item First-order differential kinematics (Jacobian): $\mathbf{\dot{x}} = J(\mathbf{q})\mathbf{\dot{q}}$, \hspace{0.2cm} $\mathbf{\dot{q}} = J^{-1}(\mathbf{q})\mathbf{\dot{x}}$
    \end{itemize}
    
	\end{itemize}
	
\end{frame}

\begin{frame}[allowframebreaks]
\frametitle{Bibliography}
\printbibliography
\end{frame}

\end{document}